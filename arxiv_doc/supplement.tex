% !TEX root = main.tex
\appendix

\clearpage
\section{Appendix: Gaussian process details}
\label{asec:GP_details}

%\vspace*{-2ex}
\paragraph*{Gaussian processes.}\hspace*{-2ex}
A GP is a stochastic process, ${f(\psi) : \psi \in \Psi }$, such that
for any finite set of elements $\psi_1, \cdots , \psi_k \in \Psi$,
the associated finite collection of random variables $f(\psi_1), \cdots, f(\psi_k)$, 
has a multivariate Gaussian distribution~\citep{b-Rasmussen2005}.

A GP $f(\psi) \sim GP(\mu(\cdot), k(\cdot,\cdot))$ can be understood as a probability distribution over arbitrary functions, with $\mu(\psi) = \mathbb{E}[f(\psi)]$ its mean function, and $k(\cdot, \cdot)$ the covariance kernel, \ie $k(\psi, \psi^\prime)=\mathbb{E}[(f(\psi)-\mu(\psi))^\top(f(\psi^\prime)-\mu(\psi^\prime))]$.
%\begin{align}
%\hspace*{-2ex}\begin{cases}
%\mu(\psi) = \mathbb{E}[f(\psi)] \;, \\
%k(\psi, \psi^\prime)=\mathbb{E}[(f(\psi)-m(\psi))^\top(f(\psi^\prime)-m(\psi^\prime))] \;.
%\end{cases}
%\end{align}
%\vspace*{-2ex}
%\paragraph*{GP model fitting.}\hspace*{-2ex}

The mean and kernel functions determine the GP function class: \ie the regularity and smoothness assumptions of the modeled data.
These are parameterized prior-functions $\mu(\cdot|\theta_{\mu})$ and $k(\cdot, \cdot|\theta_k)$,
which can be fitted to the observed data $r_{1:T} = (r_1, \cdots, r_T)$ at inputs $\psi_{1:T} = (\psi_1, \cdots, \psi_T)$.

For instance, via Type-II maximum likelihood estimation (MLE) of the GP model's hyperparameters $\theta=(\theta_{\mu}, \theta_k)$,
%\begin{equation}
$
\hat{\theta} =\argmax_{\theta} \log p\left(r_{1:T}|f(\psi_{1:T} | \theta) \right) %\;,
$,
%\label{eq:gp_hyperparameter_fit}
%\end{equation}
where the data likelihood $p(r|f (\cdot; \theta))$ is a function of the observation noise's probability distribution.
Bayesian approaches to hyperparameter selection for GP model training can also be implemented~\citep{b-Rasmussen2005}.

%\vspace*{-2ex}
\paragraph*{Gaussian process posteriors.}\hspace*{-2ex}
Given a fitted GP, posterior inference
---computing the predictive distribution of a new datapoint $\psi^\prime$ after observing $\psi_{1:T}$---
can be performed in closed form for the Gaussian observation noise case.
For example, when the noise in Equation~\eqref{eq:rewards_gp} is \iid drawn from $\epsilon_t \sim \N{\epsilon | 0, \sigma_{\epsilon}^2}$.

Formally, given a set of observations $r_{1:T}$ at inputs $\psi_{1:T}$,
the posterior distribution of $f$ is a GP with the following mean and covariance functions:
\begin{align}
&\mu_T(\psi) = k_T(\psi)^\top (K_T + \sigma_{\epsilon}^2 I)^{-1}r_{1:T} \; , \nonumber \\
&k_T(\psi, \psi^\prime) = k(\psi,\psi^\prime) - k_T(\psi)^\top (K_T + \sigma_{\epsilon}^2 I)^{-1} k_T(\psi^\prime) \;, \nonumber \\
%\sigma_T^2 &= k_T(\psi, \psi) \\
&\text{with}
\begin{cases}
k_T (\psi) = \left( k(\psi_1, \psi), \cdots, k(\psi_T, \psi)\right)^\top \;,\\
K_T = \left( k(\psi,\psi^\prime) \right)_{\forall \psi, \psi^\prime \in \psi_{1:T}} \;.
\end{cases} 
%\text{is the (positive semi-definite) kernel matrix}
\label{eq:gp_posterior}
\end{align}
These closed-form posterior inference expressions can be efficiently computed, both in exact and approximate ways~\citep{b-Rasmussen2005,ip-Pleiss2018}.
%
Posterior inference with observation noise beyond the Gaussian assumption is an active research area, with many approximate techniques available for practitioners~\citep{ic-Snelson2006,ip-Titsias2009,ip-Wilson2015,ip-Flaxman2015}.

\clearpage
\section{Appendix: Implementation and experimentation details}
\label{asec:implementation_details}

\subsection{Gaussian process}
\label{asec:implementation_details_gp}

We implement Gaussian process modules based on GPyTorch~\citep{gpytorch},
and execute all experiments with a GP process prior and GP fitting details as described in Table~\ref{tab:gp_prior}.
% !TEX root = ../main.tex
\begin{table}[!h]
	\caption{Gaussian Process prior and hyperparameters.}
	\vspace*{-2ex}
	\label{tab:gp_prior}
	\begin{center}
%		\resizebox{\columnwidth}{!}{
		\begin{tabular}{|c|c|}
			\hline
			Hyperparameter\cellcolor[gray]{0.6} & Initial Value \cellcolor[gray]{0.6} \\ \hline
\multicolumn{2}{|c|}{\cellcolor[gray]{0.9} GP Model}\\ \hline 
Mean Function & Constant \\ \hline 
Prior constant & 0 \\ \hline 
Kernel Function & Scaled RBF Kernel \\ \hline 
Prior output-scale & 1 \\ \hline 
Prior length-scale & 0.25 \\ \hline
\multicolumn{2}{|c|}{\cellcolor[gray]{0.9} Observation Model}\\ \hline 
Likelihood function & Gaussian \\ \hline 
Noise variance & 1 \\ \hline 
\multicolumn{2}{|c|}{\cellcolor[gray]{0.9} Training details}\\ \hline 
Loss function & ExactMarginalLogLikelihood \\ \hline 
train max iters & 100 \\ \hline 
loss epsilon & 0.01 \\ \hline 
\multicolumn{2}{|c|}{\cellcolor[gray]{0.9} Optimizer } \\ \hline 
optimizer & adam \\ \hline 
lr & 0.1 \\ \hline 
		\end{tabular}
%	}
	\end{center}
\end{table}

We take the most conservative approach on GP-TS prior and hyperparameter selection:
we utilize an uninformative prior, with no preference for any hyperparameter configuration.
This is the less assuming yet more challenging experimental set-up,
where we evaluate whether GP-TS can successfully learn ---without any prior knowledge--- to find good hyperparameters.

Based on bandit theory and practice,
informative priors can accelerate convergence if properly specified
(\ie when more mass is put into favorable regions of the hyperparameter space);
while slowing down convergence, if incorrectly specified
(\ie when mass is put in unfavorable regions of the space).
Evaluating how different priors affect GP-TS are experiments left as future work.

\clearpage
\subsection{RoBERTa pre-training}
\label{asec:implementation_details_roberta_pretrain}

% Computational
We pre-train all RoBERTa models as provided by~\citet{fairseq},
with the BERT-base architecture of 125M parameters, by minimizing the MLM loss with dynamic masking
in a server with 8 Tesla V100-SXM2-32GB GPUs.
%
We execute the RoBERTa pre-training procedure as described in Fairseq's RoBERTa pre-training tutorial\footnote{
	Available at \url{https://github.com/pytorch/fairseq/blob/main/examples/roberta/README.pretraining.md}
},
with specific hyperparameters as described in Table~\ref{tab:roberta_pretrain}.

The interactions for \texttt{wiki-c4} and \texttt{e-commerce} contain 1000 updates each (\ie $u=1000$), while we reduce the number of updates per-interaction to $u=500$ when pre-training with \texttt{mimic} notes.

% !TEX root = ../main.tex
\begin{table}[!h]
	\caption{RoBERTa pre-training hyperparameters.}
	\vspace*{-2ex}
	\label{tab:roberta_pretrain}
	\begin{center}
%		\resizebox{\columnwidth}{!}{
		\begin{tabular}{|c|c|}
			\hline
			Hyperparameter\cellcolor[gray]{0.6} & Value \cellcolor[gray]{0.6} \\ \hline
Architecture & RoBERTa base \\ \hline 
Task& masked lm \\ \hline 
Criterion & masked lm \\ \hline 
\multicolumn{2}{|c|}{\cellcolor[gray]{0.9} Model details}\\ \hline 
dropout & 0.1 \\ \hline 
attention-dropout & 0.1 \\ \hline 
weight-decay & 0.01 \\ \hline 
\multicolumn{2}{|c|}{\cellcolor[gray]{0.9} Training details}\\ \hline 
batch-size & 32 \\ \hline 
update-freq & 16 \\ \hline 
sample-break-mode & complete \\ \hline 
tokens-per-sample & 512 \\ \hline  
\multicolumn{2}{|c|}{\cellcolor[gray]{0.9} Optimizer } \\ \hline 
optimizer &adam \\ \hline 
adam-betas & (0.9,0.98) \\ \hline 
adam-eps & 1e-6 \\ \hline 
clip-norm & 1.0 \\ \hline 
\multicolumn{2}{|c|}{\cellcolor[gray]{0.9} Learning rate} \\ \hline 
lr &0.0005 \\ \hline 
lr-scheduler & polynomial decay \\ \hline 
linear-warmup-updates & 1000 \\ \hline 
\multicolumn{2}{|c|}{\cellcolor[gray]{0.9} Dynamic masking } \\ \hline 
mask-prob & $\rho$ \\ \hline 
leave-unmasked-prob & 0.1 \\ \hline 
random-token-prob & 0.1 \\ \hline 
		\end{tabular}
%	}
	\end{center}
\end{table}

\clearpage
\subsection{Summary statistics of the computational cost}
\label{asec:computational_overhead}

We provide in Table~\ref{tab:pretraining_compcost} summary statistics
on the execution time of GP-TS pre-training in our experiments,
as per details in Section~\ref{asec:implementation_details_roberta_pretrain}.
The per-interaction, average execution time of pre-training is:
33,316 seconds for the \texttt{wiki-c4} dataset;
37,392 seconds for the \texttt{e-commerce} data;
and 1,489 seconds for \texttt{MIMIC} notes.
It only takes about 20 seconds on average to execute GP-TS per-interaction.
Hence, the overhead is of 0.05\% for the biggest dataset, and 1\% for the smallest one.
We note that the TLM pre-training implementation of~\citet{fairseq} leverages GPU computations,
while GP-TS is executed within a single CPU ---with no GPU acceleration.

% !TEX root = supplement.tex

\begin{table}[!h]
	\caption{Per-interaction execution time of TLM pre-training and GP-TS: average time in seconds, plus-minus the standard deviation.}
	\vspace*{-2ex}
	\label{tab:pretraining_compcost} 
	\begin{center}
%		\resizebox{\columnwidth}{!}{
		\begin{tabular}{|c|c|c|c|}\hline
			 \cellcolor[gray]{0.5} &  \multicolumn{3}{c|}{Execution time in seconds \cellcolor[gray]{0.5} } \\ \cline{2-4}
			\multirow{-2}{*}{\cellcolor[gray]{0.5}Dataset} & TLM Pre-training \cellcolor[gray]{0.6} & GP-TS $\rho$ \cellcolor[gray]{0.6} & GP-TS $\psi$ \cellcolor[gray]{0.6} \\ \hline
			% wikic4
			\cellcolor[gray]{0.95} \texttt{wiki-c4} 
				& $33,316 \pm 395 \; s$ \cellcolor[gray]{0.95} &  $19 \pm 6 \; s$  \cellcolor[gray]{0.95} &  $21 \pm 6 \; s$ \cellcolor[gray]{0.95} \\  \hline 
			% medical
			\cellcolor[gray]{1.0} \texttt{mimic} 
				& $1489 \pm 46 \; s$ \cellcolor[gray]{1.0} &  $16 \pm 5 \; s$ \cellcolor[gray]{1.0} & $17 \pm 5 \; s$ \cellcolor[gray]{1.0} \\  \hline 
			% ecommerce
			\cellcolor[gray]{0.95} \texttt{e-commerce}
				& $37,392 \pm 494 \; s$ \cellcolor[gray]{0.95} & $21 \pm 3 \; s$ \cellcolor[gray]{0.95} & $23 \pm 10 \; s$ \cellcolor[gray]{0.95} \\  \hline 
		\end{tabular}
%	}
	\end{center}
\end{table}

\subsection{Summary statistics of the pre-training datasets}
\label{asec:pretraining_dataset_details}

We split each pre-training dataset into 80\%-10\%-10\% training, validation and test sets for our experiments, with summary statistics of each set provided in Table~\ref{tab:pretraining_dataset_details}.

% !TEX root = main.tex

\begin{table}[!h]
	\caption{Summary statistics of the pre-training datasets.}
	\vspace*{-2ex}
	\label{tab:pretraining_dataset_details} 
	\begin{center}
		\begin{tabular}{|c|c|c|c|}\hline
			\multicolumn{2}{c|}{Dataset \cellcolor[gray]{0.6}} & Total word count \cellcolor[gray]{0.6}  & Average words per sentence \cellcolor[gray]{0.6} \\ \hline
			% wikic4
			\cellcolor[gray]{0.95} & Training \cellcolor[gray]{0.95} &  4,517,625,794 \cellcolor[gray]{0.95} &  \; 35.9 \cellcolor[gray]{0.95} \\  
			\cellcolor[gray]{0.95} & Validation \cellcolor[gray]{0.95} & \; 735,950,955\cellcolor[gray]{0.95} & \; 35.6 \cellcolor[gray]{0.95} \\  
			\multirow{-3}{*}{\cellcolor[gray]{0.95} \texttt{wiki-c4} } & Test \cellcolor[gray]{0.95} & \; 735,571,833 \cellcolor[gray]{0.95}& \; 35.6 \cellcolor[gray]{0.95} \\ \hline 
			% medical
			\cellcolor[gray]{1.0} & Training \cellcolor[gray]{1.0} &  \; 402,720,632 \cellcolor[gray]{1.0} & 216.7 \cellcolor[gray]{1.0} \\  
			\cellcolor[gray]{1.0} & Validation \cellcolor[gray]{1.0} & \;\;\;  82,340,235 \cellcolor[gray]{1.0} & 658.7 \cellcolor[gray]{1.0} \\  
			\multirow{-3}{*}{\cellcolor[gray]{1.0} \texttt{mimic} } & Test \cellcolor[gray]{1.0} & \;\;  18,735,884  \cellcolor[gray]{1.0}& 187.3 \cellcolor[gray]{1.0} \\ \hline 
			% ecommerce
			\cellcolor[gray]{0.95} & Training \cellcolor[gray]{0.95} & 3,935,845,017 \cellcolor[gray]{0.95} & \;\; 5.6 \cellcolor[gray]{0.95} \\  
			\cellcolor[gray]{0.95} & Validation \cellcolor[gray]{0.95} & \;\;  494,802,278 \cellcolor[gray]{0.95} & \;\; 5.5 \cellcolor[gray]{0.95} \\  
			\multirow{-3}{*}{\cellcolor[gray]{0.95} \texttt{e-commerce} } & Test \cellcolor[gray]{0.95} & \;\;  482,733,197 \cellcolor[gray]{0.95} & \;\; 5.5 \cellcolor[gray]{0.95} \\ \hline 
		\end{tabular}
	\end{center}
\end{table}


\newpage
\subsection{RoBERTa fine-tuning}
\label{asec:implementation_details_roberta_fine-tune}

The specific RoBERTa hyperparameters used for the in-domain fine-tuning downstream tasks are described in Tables~\ref{tab:roberta_finetune_eclassification}--\ref{tab:roberta_finetune_medical}.

\input{./hparams/roberta_finetune_eclassification}
% !TEX root = ../main.tex
\begin{table}[!h]
	\caption{RoBERTa fine-tuning hyperparameters for the e-commerce title similarity downstream task.}
	\label{tab:roberta_finetune_esimilarity}
	\begin{center}
%		\resizebox{\columnwidth}{!}{
		\begin{tabular}{|c|c|}
			\hline
			Hyperparameter\cellcolor[gray]{0.6} & Value \cellcolor[gray]{0.6} \\ \hline
Architecture & RoBERTa base \\ \hline 
\multicolumn{2}{|c|}{\cellcolor[gray]{0.9} Task}\\ \hline 
Task& sentence prediction \\ \hline 
Criterion & sentence prediction \\ \hline 
num-classes & 2 \\ \hline 
max-positions & 512 \\ \hline
init-token & 0  \\ \hline 
separator-token & 2  \\ \hline 
\multicolumn{2}{|c|}{\cellcolor[gray]{0.9} Model details}\\ \hline 
dropout & 0.1 \\ \hline 
attention-dropout & 0.1 \\ \hline 
\multicolumn{2}{|c|}{\cellcolor[gray]{0.9} Dataset}\\ \hline 
batch-size & 32 \\ \hline 
update-freq & 1 \\ \hline 
required-batch-size-multiple & 1 \\ \hline
max-tokens & 4400 \\ \hline
skip-invalid-size-inputs-valid-test & True \\ \hline
\multicolumn{2}{|c|}{\cellcolor[gray]{0.9} Optimizer } \\ \hline 
optimizer &adam \\ \hline 
weight-decay & 0.1 \\ \hline 
adam-betas & (0.9,0.98) \\ \hline 
adam-eps & 1e-6 \\ \hline 
\multicolumn{2}{|c|}{\cellcolor[gray]{0.9} Learning rate} \\ \hline 
lr-scheduler & polynomial decay \\ \hline 
lr & 1e-5 \\ \hline 
linear-warmup-updates & 1000 \\ \hline 
max-updates & 100000 \\ \hline 
max-epoch & 10 \\ \hline
clip-norm & 0.0 \\ \hline 
		\end{tabular}
%	}
	\end{center}
\end{table}
\input{./hparams/roberta_finetune_equality}
\input{./hparams/roberta_finetune_medical}

\clearpage
\subsection{Summary statistics of the fine-tuning datasets}
\label{asec:finetuning_dataset_details}

We split each per-task fine-tuning dataset into training, development and test sets for our experiments, with summary statistics of each set provided in Table~\ref{tab:finetuning_dataset_details}.

% !TEX root = ../main.tex

\begin{table}[!h]
	\caption{Summary statistics of the fine-tuning task datasets.}
	\vspace*{-2ex}
	\label{tab:finetuning_dataset_details} 
	\begin{center}
		\begin{tabular}{|c|c|c|c|}\hline
			\multicolumn{2}{c|}{Dataset \cellcolor[gray]{0.6}} & Total sentence count \cellcolor[gray]{0.6}  & \shortstack{Average words per sentence \\ Input0 -- Input1} \cellcolor[gray]{0.6} \\ \hline
			% e-commerce title classification & similarity
			\cellcolor[gray]{0.95} & Training  \cellcolor[gray]{0.95} & 224,745 \cellcolor[gray]{0.95} & 10.9 -- 10.9\cellcolor[gray]{0.95} \\  
			\cellcolor[gray]{0.95} & Dev \cellcolor[gray]{0.95} & \;\; 6,035 \cellcolor[gray]{0.95} & 10.9 -- 10.8 \cellcolor[gray]{0.95} \\  
			\multirow{-3}{*}{\cellcolor[gray]{0.95} \shortstack{\texttt{e-commerce title} \\ \texttt{classification \& similarity}} } & Test \cellcolor[gray]{0.95} &  12,311 \cellcolor[gray]{0.95}& 10.9 -- 10.8 \cellcolor[gray]{0.95} \\ \hline 
			% e-commerce title quality
			\cellcolor[gray]{1.0} & Training \cellcolor[gray]{1.0} & 49,420 \cellcolor[gray]{1.0} & 10.6 -- NA \cellcolor[gray]{1.0} \\  
			\cellcolor[gray]{1.0} & Dev \cellcolor[gray]{1.0} & \; 2,629 \cellcolor[gray]{1.0} & 9.8 -- NA \cellcolor[gray]{1.0} \\  
			\multirow{-3}{*}{\cellcolor[gray]{1.0} \texttt{e-commerce title quality} } & Test \cellcolor[gray]{1.0} & 5,174 \cellcolor[gray]{1.0}& 9.8 -- NA \cellcolor[gray]{1.0} \\ \hline 
			% medical MLI
			\cellcolor[gray]{0.95} & Training \cellcolor[gray]{0.95} & 11,232 \cellcolor[gray]{0.95} & 15.9 -- 5.5\cellcolor[gray]{0.95} \\  
			\cellcolor[gray]{0.95} & Dev \cellcolor[gray]{0.95} & \; 1,395 \cellcolor[gray]{0.95} & 16.9 -- 5.4 \cellcolor[gray]{0.95} \\  
			\multirow{-3}{*}{\cellcolor[gray]{0.95} \texttt{medical MLI} } & Test \cellcolor[gray]{0.95} & \; 1,422 \cellcolor[gray]{0.95} & 15.4 -- 5.4 \cellcolor[gray]{0.95} \\ \hline 
		\end{tabular}
	\end{center}
\end{table}

